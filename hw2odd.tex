\documentclass[12pt]{amsart}
\usepackage{amssymb}
\usepackage[margin=1.2in]{geometry}
\usepackage{enumitem}
\setlist[enumerate,1]{label=\arabic*.}
\setlist[enumerate,2]{label=(\alph*)}
\setlist[enumerate,3]{label=(\roman*)}


\DeclareMathOperator{\perm}{P}
\DeclareMathOperator{\comb}{C}
\newcommand{\e}{\mathrm{e}}

% To have all the question numbers be odd, we use an enumerate
% environment to number the questions and, for every question after the
% first, we begin a new question with
% \stepcounter{enumi}
% \item


%--------------------------------------------------------------------
\begin{document}

\begin{center}
  \bfseries
  Xinhui Xu\\
  Homework 2, due February 16, 2018\\
  Odd numbered questions
\end{center}

\bigskip

\begin{enumerate}
\item There are three women and five men who will split up into two four-person
teams. How many ways are there to do this so that there is at least one woman
on each team?
  \begin{enumerate}[label=(\roman*)]
  \item Since the only way to have at least one woman on each team is to have a team with 1 woman and the other team with 2, we'll choose the woman to be in team A ($\binom31$ ways).
  \item We will choose the other 3 members of team A from the remaining $8-3=5$ people ($\binom53$ ways).
  \item Since team B is determined by team A, \\
    \fbox{\parbox{\linewidth}{total$=\binom31\binom53$ ways.}}
  \end{enumerate}

\stepcounter{enumi}
\item How many ways are there to distribute 5 identical apples and 6 identical pears
to 3 distinct people such that each person has at least one pear?
    \begin{enumerate}
        \item Distribute 1 pear to each of the 3 people (1 way).
        \item Distribute the 5 apples ($C(5+3-1, 3-1)=\binom72$ ways).
        \item Distribute the 3 remaining pears ($C(3+3-1, 3-1)=\binom52$ ways).
        \item \fbox{\parbox{\linewidth}{Total$=\binom72\binom52$ ways.}}
    \end{enumerate}


\stepcounter{enumi}
\item How many ways are there to distribute 16 different toys among four children?
    \begin{enumerate}
    \item Without restrictions?
        \begin{enumerate}
            \item Give each toy a label like "Child A".
            \item \fbox{\parbox{\linewidth}{Total$=4^{16}$ ways.}}
        \end{enumerate}
    \item If two children get 6 toys and two children get 2 toys?
        \begin{enumerate}
            \item Choose which 2 children get 6 toys ($C(4,2)$ ways).
            \item Choose which 12 toys the 2 children get ($C(16,12)$ ways).
            \item Choose the 6 toys for one of the children ($C(12,6)$ ways).
            \item For the two children that get 2 toys, choose 2 toys for one of the children ($C(4,2)$ ways).
            \item \fbox{\parbox{\linewidth}{Total$=\binom42\binom{16}{12}\binom{12}{6}\binom42$ ways.}}
        \end{enumerate}
    \item  With each child getting 4 toys?
        \begin{enumerate}
            \item Choose which 4 toys the first child gets ($\binom{16}{4}$ ways).
            \item Choose which 4 toys the second child gets ($\binom{12}{4}$ ways).
            \item Choose which 4 toys the third child gets ($\binom{8}{4}$ ways).
            \item \fbox{\parbox{\linewidth}{Total$=\binom{16}{4}\binom{12}{4}\binom{8}{4}$ ways.}}
        \end{enumerate}
    \end{enumerate}


\stepcounter{enumi}
\item How many integer solutions are there to x1 + x2 + x3 + x4 + x5 = 28 with:
    \begin{enumerate}
        \item $x_i\geq0: $ \begin{enumerate}
            \item Distribute 28 1's to 5 variables.
            \item \fbox{\parbox{\linewidth}{Total$=C(28+5-1,5-1)=C(32, 4)$ ways.}}
        \end{enumerate}
        \item $x_i>0: $ \begin{enumerate}
            \item Distribute a 1 to 5 variables (1 way).
            \item Distribute 28-5 1's to 5 variables ($C(28-5+5-1, 5-1)$ ways).
            \item \fbox{\parbox{\linewidth}{Total$=C(27, 4)$ ways.}}
        \end{enumerate}
    \end{enumerate}
    
\stepcounter{enumi}
\item How many arrangements of the letters in INSTRUCTOR have all of the following
properties simultaneously: (V:[I,O,U] C:[N,S,T,R,C,T,R]=[1N,2R,2T,1S,1C]) 
    \begin{enumerate}
        \item  The vowels appearing in alphabetical order.
        \item At least 2 consonants between each vowel, and
        \item Begin or end with the 2 T’s (the T’s are consecutive)
        \begin{enumerate}
            \item Choose if TT is at beginning or end (2 ways). 
            \item Put two empty slots between IOU to get I\textunderscore\textunderscore O\textunderscore\textunderscore U, and choose where the remaining empty slot is (4 ways: left of I, between IO, between OU, and right of U).
            \item We have 5 consonants [1N,2R,1S,1C], order them and place them into the slots ($P(5;2,1,1,1)=\frac{5!}{2!}$ ways).
            \item \fbox{\parbox{\linewidth}{Total$=(2)(4)(\frac{5!}{2!})$ ways.}}

        \end{enumerate}
        
    \end{enumerate}
    
\end{enumerate}

\end{document}

%%% Local Variables:
%%% mode: latex
%%% TeX-master: t
%%% End:
