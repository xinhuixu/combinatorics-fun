\documentclass[12pt]{amsart}
\usepackage{amssymb}
\usepackage{pbox}
\usepackage[margin=1.2in]{geometry}
\usepackage{enumitem}
\setlist[enumerate,1]{label=\arabic*.}
\setlist[enumerate,2]{label=(\alph*)}
\setlist[enumerate,3]{label=(\roman*)}


\DeclareMathOperator{\perm}{P}
\DeclareMathOperator{\comb}{C}
\newcommand{\e}{\mathrm{e}}




%--------------------------------------------------------------------
\begin{document}

\begin{center}
  \bfseries
  Xinhui Xu\\
  MATH225 Homework 1 (Even), due Friday February 9, 2018
\end{center}

\begin{enumerate}
\stepcounter{enumi}
\item How many ways are there to pick 2 different cards from a standard 52-card deck
such that
    \begin{enumerate}
        \item  The first card is an Ace and the second card is not a Queen?
            \begin{enumerate}
                \item Pick first card; 4 Aces in deck = 4 possibilities,
                \item Pick 2nd card; 51 cards remaining, containing 4 Queens = $(51-4)$ possibilities,
                \item \fbox{\parbox{\linewidth}{There are $(4)(51-4)$ ways.}}
            \end{enumerate}
        \item The first card is a spade and the second card is not a Queen? (Hint: Watch
out for the Queen of spades.)
            \begin{enumerate}
                \item \emph{Case 1:} Pick first card that is Queen of spades; 1 possibility,
                \item Pick 2nd card; 51 cards remaining, containing 3 Q's = $(51-3)$ possibilities,
                \item \emph{Case 2:} Pick first card that is spades but not Q; $13-1$ possibilities,
                \item Pick 2nd card; 51 cards remaining, containing 4 Q's = 51-4 possibilities,
                \item Add number of ways from case 1 and case 2 together by addition principle,
                \item \fbox{\parbox{\linewidth}{There are $1(51-3)+(13-1)(51-4)$ ways.}}
                
            \end{enumerate}
    \end{enumerate}



\stepcounter{enumi}
\item \begin{enumerate}\item How many ways are there for a man to invite some (nonempty) subset of his
10 friends to dinner?
        \begin{enumerate}
        \item Let $n$ be the number of friends he chooses to invite, which can be an integer from 1 to 10.
        \item For every group of $n$ friends there are $C(10, n)$ combinations of them,
        \item Since each combination is a unique outcome, by the addition principle,
        \item \fbox{\parbox{\linewidth}{There are $\binom{10}{10}+\binom{10}{9}+\binom{10}{8}+\binom{10}{7}+\binom{10}{6}+\binom{10}{5}+\binom{10}{4}+\binom{10}{3}+\binom{10}{2}+\binom{10}{1}$ or $\sum\limits_{i=1}^{10} \binom{10}{i}$ ways.}}
        \end{enumerate}
    \end{enumerate} 
    
\stepcounter{enumi}
\item \begin{enumerate}
    \item How many triangles are formed by
pieces of $n$ nonparallel lines, assuming
no three lines cross at a point; for example
the four lines at the right form
four triangles: $acd$, $abf$, $efd$, and $ebc$?\\
    \emph{\underline{Method one:}}
    \begin{enumerate}
        \item For all $n$ number of lines, an intersection is defined by two and only two lines crossing because we assume that no three lines can cross at a point.
        \item The number of intersections is therefore $\binom{n}{2}$.
        \item To define a triangle, we first choose a intersection point of 2 of the $n$ lines; $\binom{n}{2}$ possibilities,
        \item Choose a line, which cannot be the 2 lines that formed the chosen intersection, to complete the triangle; $(n-2)$ possibilities;

        \item Multiplying these will give us:\\
        number of possible combinations of (an intersection point) and (a line opposite that point) which forms a triangle.
        \item However, a triangle formed by choosing point, say, $a$ then line $\overline{bc}$ is the same as the triangle formed by choosing point $b$ then line $\overline{ac}$, and the triangle formed by choosing point $c$ then line $\overline{ab}$. In general,there are three point-line combinations that corresponds to the same triangle. In other words, we counted each possible triangle three times.
        \item This means that we should divide the result from (v) by 3.
        \item \fbox{\parbox{\linewidth}{Total triangles for $n\geq2$ parallel lines=\\
            \begin{displaymath}
            \frac{\binom{n}{2}(n-2)}{3}
            \end{displaymath}
        }}
        
    \end{enumerate}
    \emph{\underline{Method two:}}
    \begin{enumerate}
        \item Choose three lines from all lines. Since they are non parallel and don't intersect at the same point, each three-line combination makes a unique triangle;\\ \fbox{\parbox{\textwidth}{Total triangles = $C(n, 3)$ }}
    \end{enumerate}
\end{enumerate}

\stepcounter{enumi}
\item \begin{enumerate}
    \item How many 6-letter sequences are there with at least 3 vowels (A, E, I, O, U)?
No repetitions are allowed.
    \begin{enumerate}
        \item \emph{Case 1:} Choose 3 vowels; $\binom53$ ways, then choose where they go; $\binom63$ ways, then choose next 3 non-vowels; (21)(20)(19) ways,
        \item \emph{Case 2:} Choose 4 vowels; $\binom54$ ways, then choose where they go; $\binom64$ ways, then choose next 2 non-vowels; (21)(20) ways,
        \item \emph{Case 3:} Choose 5 vowels; $\binom55 = 1$ way, then choose where they go; $\binom65$ ways, then choose non-vowel; 21 ways.
        \item Add together by addition principle: \\ 
        \fbox{\parbox{\linewidth}{$\binom53\binom63(21)(20)(19) + \binom54\binom64(21)(20) + \binom55\binom65(21)$ sequences.}}
        
    \end{enumerate}
\end{enumerate}



\end{enumerate}

\end{document}
