\documentclass[12pt]{amsart}
\usepackage{amssymb}
\usepackage{pbox}

\usepackage{enumitem}
\setlist[enumerate,1]{label=\arabic*.}
\setlist[enumerate,2]{label=(\alph*)}
\setlist[enumerate,3]{label=(\roman*)}

\usepackage[margin=1.2in]{geometry}



\DeclareMathOperator{\perm}{P}
\DeclareMathOperator{\comb}{C}
\newcommand{\e}{\mathrm{e}}




%--------------------------------------------------------------------
\begin{document}

\begin{center}
  \bfseries
  Xinhui Xu\\
  MATH225 Homework 1 (Odd), due Friday February 9, 2018
\end{center}


\begin{enumerate}

    \item   
      \begin{enumerate}
      \item How many five-letter “words” (sequence of any five letters with repetition) are
    there?
        \begin{enumerate}
        \item Choose 1st letter; 26 possibilities,
        \item Choose the next 4 letters; each has 26 possibilities,
        \item \fbox{%
            \parbox{\linewidth}{%
                There are $26^5$ sequences.}}   
        \end{enumerate}
      \item How many with no repeated letters?
        \begin{enumerate}
            \item Choose 1st letter; 26 possibilities,
            \item Choose 2nd letter; $26-1$ pos. since 1 letter is taken,
            \item Each succeeding letter has 1 less pos. than previous,
            \item \fbox{\parbox{\linewidth}{There are $(26)(25)(24)(23)(22)=\perm(26,5)$ sequences.}}
        \end{enumerate}
            
      \end{enumerate}

\stepcounter{enumi}
\item
  \begin{enumerate}
  \item How many election outcomes are possible with 20 people each voting for
one of seven candidates (the outcome includes not just the totals but also
who voted for each candidate)?
    \begin{enumerate}
        \item Consider 20 slots representing 20 people. Pick candidate to fill the first slot; C(7,1) possibilities,
        \item Pick candidate to fill each of the remaining 19 slots; C(7,1) possibilities each,
        \item \fbox{\parbox{\linewidth}{There are $C(7,1)^{20} = 7^{20}$ ways.}}
    \end{enumerate}
  \item How many election outcomes are possible if exactly one person votes for
candidate A and exactly one person votes for candidate D?
    \begin{enumerate}
        \item Choose 1 slot to contain candidate A; C(20,1) possibilities,
        \item Choose 1 slot from the remaining 19 to contain candidate D; C(19,1) possibilities,
        \item The remaining 18 slots can be filled with candidates not A and D; C(7-2,1) possibilities each,
        \item \fbox{\parbox{\linewidth}{There are $\binom{20}{1}\binom{19}{1}\binom{7-2}{1}^{18}=20*19*5^{18}$ ways.}}
    \end{enumerate}
  \end{enumerate}


\stepcounter{enumi}
\item \begin{enumerate}
    \item How many 5-letter sequences (formed from the 26 letters in the alphabet, with
repetition allowed) contain exactly one A and exactly two Bs?
        \begin{enumerate}
            \item Choose 1 slot to contain A; C(5,1) possibilities,
            \item From the remaining 4 slots, choose 2 to contain B; C(4,2) possibilities,
            \item For each of the remaining 2 slots, choose a letter that's not A and not B; 26-2 possibilities each,
            \item \fbox{\parbox{\linewidth}{There are $\binom{5}{1}\binom{4}{2}(26-2)^2$ ways.}}
        \end{enumerate}
\end{enumerate}


\stepcounter{enumi}
\item\begin{enumerate}
    \item  A student must answer 5 out of 10 questions on a test, including at least 2 of
the first 5 questions. How many subsets of 5 questions can be answered?
    \begin{enumerate}
        \item \emph{Case 1:} Chooses 2 questions from first five; $\binom{5}{2}$ ways, then chooses 3 questions from last five; $\binom{5}{3}$ ways.
        \item \emph{Case 2:} Chooses 3 questions from first five; $\binom{5}{3}$ ways, then chooses 2 questions from last five; $\binom{5}{2}$ ways.
        \item \emph{Case 3:} Chooses 4 questions from first five; $\binom{5}{4}$ ways, then chooses 1 question from last five; $\binom{5}{1}$ ways.
        \item \emph{Case 4:} Chooses 5 questions from first five; $\binom{5}{5}$ ways, then chooses 0 question from last five; $\binom{5}{0}$ ways.
        \item Using the addition principle we can add all 4 cases:\\
        \fbox{\parbox{\linewidth}{There are $\binom52\binom53+\binom53\binom52+\binom54\binom51+\binom55\binom50$ ways.}}
    \end{enumerate}
\end{enumerate}
\stepcounter{enumi}
\item \begin{enumerate}
    \item How many ways are there to arrange n (distinct) people in a straight line so that: Mr. and Mrs.Smith are side by side;
        \begin{enumerate}
            \item Group Mr. and Mrs.Smith into one person, now there are $n-1$ people,
            \item There are 2 ways Mr. and Mrs.Smith can be side by side,
            \item and there are $(n-1)!$ ways to arrange $n-1$ distinct people,
            \item \fbox{\parbox{\linewidth}{Total: $2 * (n-1)! $ ways}}
        \end{enumerate}
                
     \item Mrs. Tucker is k positions away from the Smiths (i.e., $k - 1$ people between Mrs. Tucker and the Smiths)
        \begin{enumerate}
            \item Consider just the case where Tucker sits k positions to the right of Smiths. Group (Tucker..(k-1 people)..Smiths) into one object, and now there are $n-k-1$ objects. For example:\\
            ($n=7,k=2$) [S s {\_} T] [{\_}] [{\_}] [{\_}]\\
            $7-2-1=4$ objects, 1 (T...Ss) and 3 open slots.
            \item There are $P(4;1,3)= \frac{4!}{1!3!}=4$ ways to arrange them. Or, in general, $P((n-k-1);1,(n-k-2))=(n-k-1)$ ways.
            \item The total number of open slots is $n-3$, which means there are $(n-3)!$ arrangements of non-Tucker-Smiths.
            \item Choose how the Smiths are seated; 2 ways.
            \item Choose if Tucker sits to the left or right of the Smiths; 2 ways.
            \item \fbox{\parbox{\linewidth}{Total: $(2)(2)(n-k-1)(n-3)!$ ways}}
        \end{enumerate}

    
\end{enumerate}

\end{enumerate}

\end{document}
