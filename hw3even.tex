\documentclass[12pt]{amsart}
\usepackage{amssymb}
\usepackage[margin=1in]{geometry}
\usepackage{enumitem}
\setlist[enumerate,1]{label=\arabic*.}
\setlist[enumerate,2]{label=(\alph*)}
\setlist[enumerate,3]{label=(\roman*)}


\DeclareMathOperator{\perm}{P}
\DeclareMathOperator{\comb}{C}
\newcommand{\e}{\mathrm{e}}

% To have all the question numbers be odd, we use an enumerate
% environment to number the questions and, for every question after the
% first, we begin a new question with
% \stepcounter{enumi}
% \item


%--------------------------------------------------------------------
\begin{document}

\begin{center}
  \bfseries
  Xinhui Xu\\
  Homework 3, due February 23, 2018\\
  Even numbered questions
\end{center}

\bigskip

\begin{enumerate}
\stepcounter{enumi}
\item  Find a generating function for the number of integers between 0 and 999, 999
whose sum of digits is r.
\begin{enumerate}
    \item Each digit can be 0 to 9, there are 6 digits.
    \item \fbox{\parbox{\linewidth}{$g(x)=(1+x+x^2+x^3+...+x^9)^6$}}
\end{enumerate}
      

\stepcounter{enumi}
\item Use a generating function for modeling the number of different selections of r
hot dogs when there are five types of hot dogs.
    \begin{enumerate}
        \item We can choose from none up to infinite number of hotdogs for each type
        \item \fbox{\parbox{\linewidth}{$g(x)=(1+x+x^2+...)^5$}}
    \end{enumerate}
    
\stepcounter{enumi}
\item Find the coefficient of $x^r$ in $(x^5+x^6+x^7+...)^7$.
    \begin{enumerate}
        \item Factor out $x^5$: $x^5(1+x+x^2+...)^7=x^5(\frac{1}{1-x})^7$
        \item Expand right: $x^5(1+\binom{1+7-1}{1}x+x\binom{2+7-1}{1}x^2+...)$
        \item \fbox{\parbox{\linewidth}{$a_r=\binom{(r-5)+7-1}{1}$}}
    \end{enumerate}

\stepcounter{enumi}
\item Find the coefficient of $x^{12}$ in $x^2(1-x)^{-10}$.
\begin{enumerate}
    \item $=x^2(1+\binom{1+10-1}{1}x+x\binom{2+10-1}{1}x^2+...)$
    \item \fbox{\parbox{\linewidth}{$a_{12}=\binom{(12-2)+10-1}{1}=19$}}
\end{enumerate}


\stepcounter{enumi}
\item How many ways are there to divide five pears, five apples, five doughnuts, five
lollipops, five chocolate cats, and five candy rocks into two (unordered) piles of
15 objects each?
    \begin{enumerate}
        \item Let $g(x)=(1+x+x^2+x^3+x^4+x^5)^6$ compute $a_r$. We only need to consider one pile because the objects in it determine the objects in the other pile.
        \item Simplify: $(\frac{1-x^{5+1}}{1-x})^6=(1-x^6)^6\frac{1}{(1-x)^6}$
        \item Expand: $(1-\binom61x^6+\binom62x^{12}-\binom63x^{18}+...)(1+\binom{1+6-1}{1}x+\binom{2+6-1}{1}x^2+...)$
        \item Term $x^{15}$ looks like: $-\binom61x^6\binom{9+6-1}{1}x^9+(1)\binom{15+6-1}{1}x^{15}+\binom62x^{12}\binom{3+6-1}{1}x^3$
        \item \fbox{\parbox{\linewidth}{
        $a_{15}=(-6)(14)+(20)+\binom62(8)$ ways
        }}
    \end{enumerate}
    
\end{enumerate}
\end{document}
